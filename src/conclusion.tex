\section{Conclusion and future work}


%%%%%%%%%%%%%%%%%%%%%%%%%%%%%%%%%%%%%%%%%%%%%%%%%%%%%%%%%%%%%%%%%%%%%%%%
\begin{frame} \frametitle{Conclusion} 
{\small


    Robustness evaluation with Lazart
    \begin{itemize}
        \item filter of multi-fault attacks: equivalence and redundancy
        \item combination of fault models
        \item user accessibility, case studies
        \item[] 
    \end{itemize}

    Countermeasures evaluation
    \begin{itemize}
        \item isolation analysis
        \item placement algorithms
        \item[] $\rightarrow$ gives strong guarantees, even if the trace set is incomplete 
        \item[] $\rightarrow$ allows combination of fault model in multiple faults
        \item detector optimization algorithm
        \item[] $\rightarrow$ up to 80\% of \texttt{detectors} removed
        \item[]
    \end{itemize}

     
{\tiny
        \textbf{Publications:}
        \begin{itemize}
            \item \textit{Combining Static Analysis and Dynamic Symbolic Execution in a Toolchain to detect Fault Injection Vulnerabilities}
            \item[] Guilhem Lacombe, David Féliot, Etienne Boespflug and Marie-Laure Potet, \textit{Journal of Cryptography Engineering 2023}
            \item[]
            \item \textit{Combining Static Analysis and Dynamic Symbolic Execution in a Toolchain to detect Fault Injection Vulnerabilities}
            \item[] Guilhem Lacombe, David Féliot, Etienne Boespflug and Marie-Laure Potet, \textit{PROOFS Workshop 2021}
            \item[]
            \item \textit{Countermeasures Optimization in Multiple Fault-Injection Context}
            \item[] Etienne Boespflug, Cristian Ene, Marie-Laure Potet and Laurent Mouniter, \textit{FDTC 2020}
        \end{itemize}
}
\vfill
}
\end{frame}

%%%%%%%%%%%%%%%%%%%%%%%%%%%%%%%%%%%%%%%%%%%%%%%%%%%%%%%%%%%%%%%%%%%%%%%%
\begin{frame} \frametitle{Future Works} 
{\tiny


    Tools:
    \begin{itemize}
        \item Extension of fault models in Lazart
        \item Extension of automated countermeasures in Lazart
        \item Validate contribution on more example programs
        \item Combination with static analysis
        \item "fault-aware" dynamic-symbolic execution engine
        \item [] 
    \end{itemize}
    
    \onslide<2->{
    Countermeasure placement:
    \begin{itemize}
        \item Study of countermeasures without IP granularity
        \item Study of countermeasures propagating states (SSCF, Swift...)
        \item[] $\rightarrow$ may require to consider two isolation analysis cases: sane CM's inputs and corrupted CM's inputs
        \item Study of more complex CFG fault models
        \item [] $\rightarrow$ requires to take into account the several entry and output points of the protection scheme
        \item Extension of notion of \textit{adequation}, \textit{perfection} of CMs and \textit{protectability} of fault models
        \item[] 
    \end{itemize}
    }
    
    \onslide<3->{
    Countermeasure optimization:
    \begin{itemize}
        \item Switch to Lazart 4
        \item Fully symbolic version (relax constraints on detectors)
        \item[] $\rightarrow$ internal states may be difficult to consider
        \item Study of other countermeasures
        \item[] 
    \end{itemize}
    }

\vfill
}
\end{frame}

\begin{comment}
%%%%%%%%%%%%%%%%%%%%%%%%%%%%%%%%%%%%%%%%%%%%%%%%%%%%%%%%%%%%%%%%%%%%%%%%
\begin{frame} \frametitle{Future Works} 
{\tiny


\begin{columns}
        
    \begin{column}{0.3\textwidth}
    Implementation:
    \begin{itemize}
        \item Extension of fault model in Lazart
        \item Extension of automated countermeasures in Lazart
        \item Validate contribution on more example program
        \item Binary level implementation
        \item [] 
        \item[] 
        \item[] 
        \item[] 
    \end{itemize}
    \end{column}
    
    \begin{column}{0.4\textwidth}
    \onslide<2->{
    Countermeasure placement
    \begin{itemize}
        \item Study of countermeasures without IP granularity
        \item Study of countermeasures propagating states 
        \item[] $\rightarrow$ may require to consider two isolation analysis cases: sane CM's inputs and corrupted CM's inputs
        \item Study of more complex fault models
        \item [] $\rightarrow$ requires to take into account the several entry and output points of the protection scheme
        \item Study of incomplete catalogs
        \item[] 
    \end{itemize}
    }
    \end{column}
    
    \begin{column}{0.4\textwidth}
    \onslide<3->{
    Countermeasure optimization
    \begin{itemize}
        \item Fully symbolic version (relax constraints on detectors)
        \item How to protect detectors with other fault models ?
        \item Over-approximation approaches 
        \item[] 
        \item[] 
        \item[] 
        \item[] 
        \item[] 
    \end{itemize}
    }
    \end{column}
    
        
\end{columns}
\vspace{0.2cm}

\vfill
}
\end{frame}
    
\end{comment}


%%%%%%%%%%%%%%%%%%%%%%%%%%%%%%%%%%%%%%%%%%%%%%%%%%%%%%%%%%%%%%%%%%%%%%%%
\begin{frame}[fragile, noframenumbering] \frametitle{The End} 
\vspace{1cm}
\begin{center}
    {\large Thanks for watching}
\end{center}
\vfill
\end{frame}

\begin{comment}

\subsection{Conclusion}

%%%%%%%%%%%%%%%%%%%%%%%%%%%%%%%%%%%%%%%%%%%%%%%%%%%%%%%%%%%%%%%%%%%%%%%%
\begin{frame}[fragile] \frametitle{Conclusion} 
\begin{itemize}
    \item A methodology to \textit{optimize} program protected by \textit{side-effect free} detectors (up to 80\% of \texttt{detectors} removed)
    \item[]
    \item Only one program exploration $\rightarrow$ realistic analysis time for real world programs
    \item[]
    \item The methodology is generic regarding to the analysis level and trace generation method
    
\end{itemize}
\vfill
\end{frame}



\subsection{Conclusion}
%%%%%%%%%%%%%%%%%%%%%%%%%%%%%%%%%%%%%%%%%%%%%%%%%%%%%%%%%%%%%%%%%%%%%%%%
\begin{frame} \frametitle{Conclusion} 
{\small

\begin{itemize}
    \item Robustness of placement depend on the property of the catalog
    \item[]
    \item P' is guaranteed as robust if the required protection coefficients (K) are available
    \item[] $\rightarrow$ if not, attack traces on P' are known
    \item []$\rightarrow$ more robust than P even if trace set is incomplete
    \item[]
    \item Optimal placement is guaranteed with ILP
    \item[]
    \item Protection weight: $rep < bloc < min < atk < naive$ 
    \item[]
\end{itemize}

            \begin{table}[h]
                {\tiny
                \begin{center}%\setlength\tabcolsep{1.5pt}
\begin{tabular}{l|l|ll|l|lll}
Algorithme & Type & \multicolumn{2}{l|}{Guarantees $P'$} & Complexity & \multicolumn{3}{l}{Required analysis} \\
 &  & Robust & Optimal &  & AA & Red & HS \\
 \hline
naive & syst. & \checkmark & - & $O(t)$ & \checkmark & - & - \\
atk & syst. & \checkmark & - & $O(t)$ & \checkmark & - & - \\
min & syst. & \checkmark & - & $O(t)$ & \checkmark & \checkmark & - \\
bloc-h & bloc & \checkmark & - & $O(t)$ & \checkmark & \checkmark & \checkmark \\
rep-opt & distributed & \checkmark & \checkmark & NP-Complete & \checkmark & \checkmark & - \\
\end{tabular}
                \end{center} 
                }
                \end{table}
    

\begin{itemize}
    \item Placement algorithm is slow compared to DSE and redundancy analysis
    \item[] $\rightarrow$ event with optimal algorithm and ILP (1-fault attacks)
\end{itemize}
}
\vfill
\end{frame}


%%%%%%%%%%%%%%%%%%%%%%%%%%%%%%%%%%%%%%%%%%%%%%%%%%%%%%%%%%%%%%%%%%%%%%%%
\begin{frame} \frametitle{Future Works} 
{\small



\begin{itemize}
    \item Study of countermeasures without IP granularity
    \item[]
    \item Study of countermeasures propagating states 
    \item[] $\rightarrow$ may require to consider two isolation analysis cases: sane CM's inputs and corrupted CM's inputs
    \item[]
    \item Study of more complex fault models
    \item [] $\rightarrow$ requires to take into account the several entry and output points of the protection scheme
    \item[]
    \item Study of incomplete catalogs
\end{itemize}

\vfill
}
\end{frame}
\end{comment}
